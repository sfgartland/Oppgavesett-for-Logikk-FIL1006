% !TeX root = ..\main.tex
\documentclass[../main.tex]{subfiles}

\setcounter{chapter}{1}
\setcounter{section}{5}

\begin{document}

\section{Uintuitive deduksjoner}

%% Litt rar, kort men uintuitiv
\bigskip
\subsection{Øvelse} \label{ex:unintuitive:1}
Bevis at fra \(p\) kan man konkludere \(q \to p\).

%% Veldig uintiutiv men interessant
\bigskip
\subsection{Øvelse} \label{ex:unintuitive:2}
Bevis at fra \(\lnot p\) kan man konkludere \(p \to q\).


\newpage
\subsection{Løsningsforslag}


%% Litt rar, kort men uintuitiv
\bigskip
\subsubsection{Øvelse \ref{ex:unintuitive:1}} \label{ex:unintuitive:1:solution}
Bevis at fra \(p\) kan man konkludere \(q \to p\).

\subsubsection*{Bevis:}
\[
    \begin{nd}
        \hypo{1}{p}
        \have{2}{p} \r{1}
        \open
        \hypo{3}{q}
        \have{4}{p} \r{2}
        \close
        \have{5}{q \to p} \ii{3-4}
    \end{nd}
\]


%% Veldig uintiutiv men interessant
\bigskip
\subsubsection{Øvelse \ref{ex:unintuitive:2}} \label{ex:unintuitive:2:solution}
Bevis at fra \(\lnot p\) kan man konkludere \(p \to q\).

\subsubsection*{Bevis:}
\[
\begin{nd}
  \hypo{1}{\lnot p}
  \open
    \hypo{2}{p}
    \open
    \hypo{3}{\lnot q}
    \have{4}{\lnot p} \r{1}
    \have{5}{p} \r{2}
    \close
    \have{6} {q} \ni{3-5}
  \close
  \have{5}{p \to q} \ii{2-4}
\end{nd}
\]

\end{document}