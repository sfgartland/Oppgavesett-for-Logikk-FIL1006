% !TeX root = ..\main.tex
\documentclass[../main.tex]{subfiles}

\setcounter{chapter}{1}
\setcounter{section}{1}

\begin{document}

\section{Oppvarmingsoppgaver}

\bigskip
\subsection{Øvelse} \label{ex:oppvarming:1}
Vis at fra \(p\) og \(q\) kan man konkludere \(p \land q\).


\bigskip
\subsection{Øvelse} \label{ex:oppvarming:2}
Vis at fra \(p \land q\) kan man konkludere \(p\).



\bigskip
\subsection{Øvelse} \label{ex:oppvarming:3}
Vis at fra \(p\) kan man konkludere \(p \lor q\).


\bigskip
\subsection{Øvelse} \label{ex:oppvarming:4}
Vis at \(p \to p\).


\bigskip
\subsection{Øvelse} \label{ex:oppvarming:5}
Vis at fra \(p\) og \(p \to q\) kan man konkludere \(q\).




\newpage
\subsection{Løsningsforslag}

% Øvelse 1: Konjunksjonsintroduksjon (∧I)
\bigskip
\subsubsection{Øvelse \ref{ex:oppvarming:1}} \label{ex:oppvarming:1:solution}
Vis at fra \(p\) og \(q\) kan man konkludere \(p \land q\).

\subsubsection*{Bevis:}
\[
\begin{nd}
  \hypo{1}{p}
  \hypo{2}{q}
  \have{3}{p} \r{1}
  \have{4}{q} \r{2}
  \have{5}{p \land q} \ai{3,4}
\end{nd}
\]


% Øvelse 2: Konjunksjonseleminering (∧E)
\bigskip
\subsubsection{Øvelse \ref{ex:oppvarming:2}} \label{ex:oppvarming:2:solution}
Vis at fra \(p \land q\) kan man konkludere \(p\).

\subsubsection*{Bevis:}
\[
\begin{nd}
  \hypo{1}{p \land q}
  \have{2}{p} \ae{1}
\end{nd}
\]


% Øvelse 3: Disjunksjonsintroduksjon (∨I)
\bigskip
\subsubsection{Øvelse \ref{ex:oppvarming:3}} \label{ex:oppvarming:3:solution}
Vis at fra \(p\) kan man konkludere \(p \lor q\).

\subsubsection*{Bevis:}
\[
\begin{nd}
  \hypo{1}{p}
  \have{2}{p \lor q} \oi{1}
\end{nd}
\]


% Øvelse 5: Impliksjonsintroduksjon (→I)
\bigskip
\subsubsection{Øvelse \ref{ex:oppvarming:4}} \label{ex:oppvarming:4:solution}
Vis at \(p \to p\).

\subsubsection*{Bevis:}
\[
\begin{nd}
  \open
    \hypo{1}{p}
    \have{2}{p} \r{1}
  \close
  \have{3}{p \to p} \ii{1-2}
\end{nd}
\]


% Øvelse 6: Impliksjonseleminering (→E)
\bigskip
\subsubsection{Øvelse \ref{ex:oppvarming:5}} \label{ex:oppvarming:5:solution}
Vis at fra \(p\) og \(p \to q\) kan man konkludere \(q\).

\subsubsection*{Bevis:}
\[
\begin{nd}
  \hypo{1}{p}
  \hypo{2}{p \to q}
  \have{3}{q} \ie{1,2}
\end{nd}
\]



\end{document}
