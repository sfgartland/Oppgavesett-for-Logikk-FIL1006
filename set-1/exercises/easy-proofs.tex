% !TeX root = ..\main.tex
\documentclass[../main.tex]{subfiles}

\setcounter{chapter}{1}
\setcounter{section}{2}


\begin{document}

\section{Nivå 1}

\bigskip
\subsection{Øvelse} \label{ex:easy:1}
Vis at \(p \to (p \lor q)\).


% Øvelse 7: Negasjonsintroduksjon (¬I)
\bigskip
\subsection{Øvelse} \label{ex:easy:2}
Vis at fra \(p \to q\) og \(\lnot q\) kan man konkludere \(\lnot p\).


\bigskip
\subsection{Øvelse}  \label{ex:easy:3}
Vis at fra \(p \to q\), kan man konkludere \(\neg q \to \neg p\).


%% Fin intro
\bigskip
\subsection{Øvelse} \label{ex:easy:4}
Bevis at fra \(p \to q\) og \(q \to r\), kan man konkludere \(p \to r\).


% Øvelse 4: Disjunksjonseleminering (∨E)
\bigskip
\subsection{Øvelse} \label{ex:easy:5}
Vis at fra \(p \lor q\), \(p \to r\) og \(q \to r\) kan man konkludere \(r\).









\newpage
\subsection{Løsningsforslag}

\bigskip
\subsubsection{Øvelse \ref{ex:easy:1}} \label{ex:easy:1:solution}
Vis at \(p \to (p \lor q)\).

\subsubsection*{Bevis}
\[
\begin{nd}
  \open
  \hypo {1} {p}
  \have {2} {p \lor q} \oi{1}
  \close
  \have {3} {p \to (p \lor q)} \ii{1-3}
\end{nd}
\]

% Øvelse 7: Negasjonsintroduksjon (¬I)
\bigskip
\subsubsection{Øvelse \ref{ex:easy:2}} \label{ex:easy:2:solution}
Vis at fra \(p \to q\) og \(\lnot q\) kan man konkludere \(\lnot p\).

\subsubsection*{Bevis}
\[
\begin{nd}
  \hypo{1}{p \to q}
  \hypo{2}{\lnot q}
  \open
    \hypo{3}{p}
    \have{4}{q} \ie{3,1}
    \have{5}{\lnot q} \r{2}
  \close
  \have{6}{\lnot p} \ni{3-5}
\end{nd}
\]

\bigskip
\subsubsection*{Øvelse \ref{ex:easy:3}} \label{ex:easy:3:solution}
Vis at fra \(p \to q\), kan man konkludere \(\neg q \to \neg p\).

\subsubsection*{Bevis}
\[
\begin{nd}
  \hypo {1} {p \to q} 
  \open
  \hypo {2} {\neg q}
  \open
  \hypo {3} {p} 
  \have {4} {q} \ie{1, 3}
  \have {5} {\neg q} \r{2}
  \close
  \have {6} {\neg p} \ni{3-5}
  \close
  \have {7} {\neg q \to \neg p} \ii{2-6}
\end{nd}
\]


%% Fin intro
\bigskip
\subsubsection{Øvelse \ref{ex:easy:4}} \label{ex:easy:4:solution}
Bevis at fra \(p \to q\) og \(q \to r\), kan man konkludere \(p \to r\).

\subsubsection*{Bevis}
\[
    \begin{nd}
        \hypo{1}{p \to q}
        \hypo{2}{q \to r}
        \open
        \hypo{3}{p}
        \have{4}{q} \ie{3,1}
        \have{5}{r} \ie{4,2}
        \close
        \have{6}{p \to r} \ii{3-5}
    \end{nd}
\]

% Øvelse 4: Disjunksjonseleminering (∨E)
\bigskip
\subsubsection{Øvelse \ref{ex:easy:5}} \label{ex:easy:5:solution}
Vis at fra \(p \lor q\), \(p \to r\) og \(q \to r\) kan man konkludere \(r\).

\subsubsection*{Bevis}
\[
\begin{nd}
  \hypo{1}{p \lor q}
  \hypo{2}{p \to r}
  \hypo{3}{q \to r}
  \open
    \hypo{4}{p}
    \have{5}{r} \ie{4,2}
  \close
  \open
    \hypo{6}{q}
    \have{7}{r} \ie{6,3}
  \close
  \have{8}{r} \oe{1,4-5,6-7}
\end{nd}
\]


\end{document}
